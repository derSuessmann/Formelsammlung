% !TeX root=abs_formeln.tex

\subsection{Photoeffekt}

\subsubsection{Maximale Energie der Photoelektronen}
\begin{equation}\label{eq:photoeffekt:maximale:energie:elektron}
W_{max} = e \cdot U_{max} \quad W_{max} = h \cdot f - W_A
\end{equation}

\subsubsection{Grenzfrequenz der Elektronenablösung}
\begin{equation}\label{eq:photoeffekt:grenzfrequenz}
f_{gr} = \frac{W_A}{h}
\end{equation}

\subsubsection{Photostrom}
\begin{equation}\label{eq:photoeffekt:photostrom}
I_{Ph} = Z \cdot \frac{e}{t}
\end{equation}

\subsection{Umkehrung des Photoeffekts}
\begin{equation}\label{eq:umgekehrter:photoeffekt:energie}
W_{El} = e \cdot U = h \cdot f
\end{equation}

\begin{equation}\label{eq:umgekehrter:photoeffekt:maximale:frequenz}
f_{max} = \frac{c}{\lambda_{min}}
\end{equation}

\begin{equation}\label{eq:umgekehrter:photoeffekt:energiebilanz}
h \cdot f_{max} = e \cdot U
\end{equation}

\subsection{Masse-Energie-Äquivalent}
\begin{equation}\label{eq:masse:energie:aequivalent}
W_0 = m_0 \cdot c^2
\end{equation}

\subsection{Masse der Photonen}
\begin{equation}\label{eq:photon:masse}
m = \frac{W}{c^2} = \frac{h \cdot f}{c^2}
\end{equation}

\subsection{Impuls der Photonen}
\begin{equation}\label{eq:photon:impuls}
p = m \cdot v = \frac{h \cdot f}{c} = \frac{h}{\lambda}
\end{equation}

\subsection{Paarbildung}
Photon $\rightarrow$ Elektron + Positron

\subsubsection{Energieerhaltung}
\begin{equation}\label{eq:paarbildung:energieerhaltung}
h \cdot f = 2\cdot m_e \cdot c^2 + 2 \cdot W_{kin} \ge
\SI{1.02}{\mega\electronvolt}
\end{equation}

\subsubsection{Massenerhaltung}
\begin{equation}\label{eq:paarbildung:massenerhaltung}
\frac{h\cdot f}{c^2} = \frac{2 \cdot W_{kin}}{c^2} + 2 \cdot m_e
\end{equation}

\subsubsection{Impulserhaltung}
\begin{equation}\label{eq:paarbildung:impulserhaltung}
\frac{h \cdot f}{c} = 2 \cdot m_e \cdot v < 2 \cdot m_e \cdot c \le
\frac{h \cdot f}{c}
\end{equation}

\subsection{Zerstrahlung}
Elektron + Positron $\rightarrow$ 2 Photonen

\subsubsection{Energieerhaltung}
\begin{equation}\label{eq:zerstrahlung:energieerhaltung}
2\cdot m_e \cdot c^2 = 2 \cdot h \cdot f =
\SI{1.02}{\mega\electronvolt}
\end{equation}

\subsubsection{Massenerhaltung}
\begin{equation}\label{eq:zerstrahlung:massenerhaltung}
2 \cdot m_e = \frac{2 \cdot h\cdot f}{c^2}
\end{equation}

\subsubsection{Impulserhaltung}
\begin{equation}\label{eq:zerstrahlung:impulserhaltung}
0 = \frac{h \cdot f}{c} + \left( - \frac{h \cdot f}{c} \right) 
\end{equation}

\subsection{Compton-Effekt}
\begin{equation}\label{eq:compton:effekt}
\Delta \lambda = \lambda' - \lambda = \lambda_C \cdot \left(1 - \cos\beta
\right)
\end{equation}

\begin{equation}\label{eq:compton:wellenlaenge}
\lambda_C = \frac{h}{m_e \cdot c} = \SI{2.4}{\pico\meter}
\end{equation}

\subsection{Photon als Quantenobjekt}
\begin{equation}\label{eq:photonquantenobjekt}
\Psi_{Res} = \Psi_1 + \Psi_2 \quad \left|\Psi_{Res}\right|^2 = \left|\Psi_1 +
\Psi_2\right|
\end{equation}
Antreffwahrscheinlichkeit: $\left|\Psi\right|^2$

\subsection{De-Broglie-Wellenlänge}
\begin{equation}\label{eq:debroglie}
\lambda_B = \frac{h}{p}
\end{equation}

