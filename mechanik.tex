% !TeX root=abs_formeln.tex

\subsection{Geradlinige, gleichförmige Bewegung}
\begin{equation}\label{eq:mechanik:geradlinige:gleichfoermige:bewegung:v}
v = \frac{\Delta s}{\Delta t}
\end{equation}

\subsection{Geradlinige, gleichmäßig beschleunigte Bewegung}
\begin{equation}\label{eq:mechanik:geradlinige:gleichmaessig:beschleunigte:bewegung:a}
a = \frac{\Delta v}{\Delta t}
\end{equation}

\begin{equation}\label{eq:mechanik:geradlinige:gleichmaessig:beschleunigte:bewegung:s:v}
\Delta s = \frac{(v_2 + v_1) \cdot \Delta t}{2}
\end{equation}  

\begin{equation}\label{eq:mechanik:geradlinige:gleichmaessig:beschleunigte:bewegung:s:a}
s(t) = \frac{1}{2}\cdot a\cdot t^2 + v_0 \cdot t + s_0
\end{equation}

\subsection{Grundgleichung der Mechanik (Newtons Grundgesetz)}
\begin{equation}\label{eq:mechanik:grundgleichung}
 F = m \cdot a
\end{equation}

\subsection{Gewichtskraft}
\begin{equation}\label{eq:mechanik:gewichtskraft}
 F_G = m \cdot g
\end{equation}

\subsection{Hookesches Gesetz}
\begin{equation}\label{eq:mechanik:hookesches:gesetz}
 F = D \cdot s
\end{equation}

\subsection{Schiefe Ebene}
\begin{equation}\label{eq:mechanik:schiefe:ebene:hangabtriebskraft}
 F_H = F_G \cdot \sin \alpha
\end{equation}

\begin{equation}\label{eq:mechanik:schiefe:ebene:normalkraft}
 F_N = F_G \cdot \cos \alpha
\end{equation}

\subsection{Reibung}
\begin{equation}\label{eq:mechanik:reibung:ordnung}
 F_h > F_{gl} > F_{roll}
\end{equation}

\begin{equation}\label{eq:mechanik:gleitreibung:kraft}
 F_{gl} = f_{gl} \cdot F_N 
\end{equation}

\begin{equation}\label{eq:mechanik:haftreibung:maximale:kraft}
 F_{h,max} = f_{h} \cdot F_N 
\end{equation}

\subsection{Bremsverzögerung}
\begin{equation}\label{eq:mechanik:gleitreibung:bremsverzoegerung}
 \left| a \right| = f_{gl} \cdot g
\end{equation}

\begin{equation}\label{eq:mechanik:haftreibung:bremsverzoegerung}
 \left| a \right| = f_{h} \cdot g
\end{equation}

\subsection{Zentripetalkraft}
\begin{equation}\label{eq:mechanik:zentripetalkraft}
 F_z = \frac{m \cdot v^2}{r}
\end{equation}

\subsection{Energieerhaltung}
\begin{align}
\label{eq:mechanik:energieerhaltung}
 W_L + W_B + W_{Sp}  &= \text{konst.}\\
\label{eq:mechanik:lageenergie}
 W_L  &= m \cdot g \cdot h \\
\label{eq:mechanik:bewegungsenergie}
 W_B &= \frac{1}{2}\cdot m \cdot v^2 \\
\label{eq:mechanik:spannenergie}
W_{Sp} &=  \frac{1}{2} \cdot D \cdot s^2
\end{align}

\subsection{Energie / Arbeit}
\begin{equation}\label{eq:mechanik:energie}
 W = F_s \cdot s
\end{equation}

\subsection{Leistung}
\begin{equation}\label{eq:mechanik:leistung}
 P = \frac{\Delta W}{\Delta t}
\end{equation}

\subsection{Impuls}
\begin{equation}\label{eq:mechanik:impuls}
 p = m \cdot v
\end{equation}

\subsection{Impulserhaltung}
\begin{equation}\label{eq:mechanik:impulserhaltung}
 m_1 \cdot u_1 + m_2 \cdot u_2 = m_1 \cdot v_1 + m_2 \cdot v_2 
\end{equation}