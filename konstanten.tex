% !TeX root=abs_formeln.tex

\begin{align}
\label{eq:konstanten:fallbeschleunigung}
g &= \SI{9.81}{\meter\per\second\squared}\\
\label{eq:konstanten:lichtgeschwindigkeit}
c &= \SI{3d8}{\meter\per\second}\\
\label{eq:konstanten:elementarladung}
e &= \SI{1.6021773d-19}{\coulomb}\\
\label{eq:konstanten:elektrische:feldkonstante}
\varepsilon_0 &= \SI{8.85d-12}{\coulomb\per\volt\per\meter}\\
\label{eq:konstanten:magnetische:feldkonstante}
\mu_0 &= \SI{1.257d-6}{\volt\second\per\ampere\per\meter}\\
\label{eq:konstanten:planck:konstante}
h &= \SI{6.626d-34}{\joule\second}\\
\label{eq:konstanten:atomare:masseneinheit}
u &= \SI{1.66054d-27}{\kilogram}\\
\label{eq:konstanten:avogadro:konstante}
N_A &= \SI{6.022d23}{\per\mol}\\
\label{eq:konstanten:masse:elektron}
m_e &= \SI{9.1d-31}{\kilogram}\\
\label{eq:konstanten:masse:neutron}
m_n &= \SI{1.675d-27}{\kilogram}\\
\label{eq:konstanten:masse:proton}
m_p &= \SI{1.673d-27}{\kilogram}\\
\label{eq:konstanten:allgemeine:gaskonstante}
R &= \SI{8.31}{\joule\per\mol\per\kelvin}\\
\label{eq:konstanten:rydberg:konstante}
R_\infty &= \SI{1.097e7}{\per\meter}
\end{align}