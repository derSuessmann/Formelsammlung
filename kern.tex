% !TeX root=abs_formeln.tex

\subsection{Energie der Elektronen in der Atomhülle}
\begin{equation}\label{eq:potentielle:energie:elektron:atomhuelle}
W_p = - \frac{1}{4\cdot\pi\cdot\varepsilon_0}\cdot \frac{\left(Z\cdot e\right)\cdot e}{r}
\end{equation}

\begin{equation}\label{eq:kinetische:energie:elektron:atomhuelle}
W_k = \frac{1}{8\cdot\pi\cdot\varepsilon_0}\cdot \frac{\left(Z\cdot e\right)\cdot e}{r}
\end{equation}

\begin{equation}\label{eq:gesamt:energie:elektron:atomhuelle}
W_{ges} = - \frac{1}{8\cdot\pi\cdot\varepsilon_0}\cdot \frac{\left(Z\cdot e\right)\cdot e}{r}
\end{equation}

\subsection{Erstes Bohr-Postulat}
\begin{equation}\label{eq:erstes:bohr:postulat:1}
2\cdot\pi\cdot m_e \cdot r_n \cdot v_r = n \cdot h
\end{equation}
oder
\begin{equation}\label{eq:erstes:bohr:postulat:2}
2\cdot\pi\cdot m_e \cdot r^2_n \cdot \omega_r = n \cdot h
\end{equation}

\subsection{Zweites Bohr-Postulat}
\begin{equation}\label{eq:zweites:bohr-postulat}
h\cdot f = W_2 - W_1
\end{equation}

\subsection{Moseley-Gesetz}
\begin{equation}\label{eq:moseley:gesetz}
\begin{split}
N_{K_\alpha} &= R_\infty \left(Z-1\right)^2\left(\frac{1}{1^2}-\frac{1}{2^2}\right)\\
 &= \frac{3}{4}R_\infty\left(Z-1\right)^2
\end{split}
\end{equation}

\subsection{Zerfallsgesetz}
\begin{equation}\label{eq:zerfallsgesetz:zerfallskonstante}
N(t) = N_0 \cdot e^{-\lambda \cdot t}
\end{equation}

\begin{equation}\label{eq:zerfallsgesetz:halbwertszeit}
N(t) = N_0 \cdot 2^{-\frac{t}{T_{1/2}}}
\end{equation}

\subsection{Zerfallskonstante}
\begin{equation}\label{eq:zerfallskonstante}
\lambda = \frac{\ln 2}{T_{1/2}}
\end{equation}

\subsection{Aktivität}
\begin{align}
\label{eq:aktivitaet}
A &= \frac{\Delta N}{\Delta t} \\
\label{eq:aktivitaet:t}
A(t) &= \lambda \cdot N(t) = A_0 \cdot
e^{-\lambda \cdot t}
\end{align}
