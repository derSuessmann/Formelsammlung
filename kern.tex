% !TeX root=abs_formeln.tex

\subsection{Abschätunzung der Kerngröße (Rutherford)}
\begin{equation}\label{eq:kerngroesse:rutherford}
\frac{1}{2}\cdot m\cdot v^2 = \frac{1}{4\cdot \pi \cdot \varepsilon_0}\cdot\frac{Z\cdot e\cdot 2\cdot e}{b}
\end{equation}

\subsection{Energie der Elektronen in der Atomhülle}
\begin{equation}\label{eq:potentielle:energie:elektron:atomhuelle}
W_p = - \frac{1}{4\cdot\pi\cdot\varepsilon_0}\cdot \frac{\left(Z\cdot e\right)\cdot e}{r}
\end{equation}

\begin{equation}\label{eq:kinetische:energie:elektron:atomhuelle}
W_k = \frac{1}{8\cdot\pi\cdot\varepsilon_0}\cdot \frac{\left(Z\cdot e\right)\cdot e}{r}
\end{equation}

\begin{equation}\label{eq:gesamt:energie:elektron:atomhuelle}
W_{ges} = - \frac{1}{8\cdot\pi\cdot\varepsilon_0}\cdot \frac{\left(Z\cdot e\right)\cdot e}{r}
\end{equation}

\subsection{Erstes Bohr-Postulat (Bahndrehimpuls)}
\begin{equation}\label{eq:erstes:bohr:postulat:1}
L = r \cdot m \cdot v = n \cdot \frac{h}{2\cdot\pi} \quad n = 1, 2, 3, \ldots
\end{equation}

\subsection{Zweites Bohr-Postulat (Frequenzbedingung)}
\begin{equation}\label{eq:zweites:bohr-postulat}
h\cdot f = E_m - E_n = \Delta E
\end{equation}

\subsection{Frequenz des Photons}
\begin{equation}\label{eq:bohr:frequenz:photon}
\begin{split}
f &= R \left(\frac{1}{n^2}-\frac{1}{m^2}\right)
\end{split}
\end{equation}

\subsubsection{Rydberg-Frequenz}
\begin{equation}\label{eq:rydberg:frequenz}
R = \frac{m_e \cdot e^4}{8 \cdot \varepsilon_0^2 \cdot h^3} = \SI{3.2898e15}{\hertz} 
\end{equation}

\subsection{Zerfallsgesetz}
\begin{equation}\label{eq:zerfallsgesetz:zerfallskonstante}
N(t) = N_0 \cdot e^{-\lambda \cdot t}
\end{equation}

\begin{equation}\label{eq:zerfallsgesetz:halbwertszeit}
N(t) = N_0 \cdot 2^{-\frac{t}{T_{1/2}}}
\end{equation}

\subsection{Zerfallskonstante}
\begin{equation}\label{eq:zerfallskonstante}
\lambda = \frac{\ln 2}{T_{1/2}}
\end{equation}

\subsection{Aktivität}
\begin{align}
\label{eq:aktivitaet}
A &= \frac{\Delta N}{\Delta t} \\
\label{eq:aktivitaet:t}
A(t) &= \lambda \cdot N(t) = A_0 \cdot
e^{-\lambda \cdot t}
\end{align}
